\documentclass[12pt]{article}

% ---------------------------------------------------------------------------
% Packages
% ---------------------------------------------------------------------------
\usepackage[utf8]{inputenc}        % For UTF-8 encoding
\usepackage[T1]{fontenc}           % For better encoding of special characters
\usepackage{amsmath,amssymb}       % For advanced math typesetting
\usepackage{graphicx}              % For including images
\usepackage{hyperref}              % For clickable URLs and cross-references
\usepackage[backend=biber]{biblatex} % For bibliography management
\addbibresource{references.bib}    % Example references file

\usepackage{lipsum}                % For sample text
\usepackage{babel}                 % For language-specific typographical rules
\usepackage{booktabs}              % For nicer tables
\usepackage{enumitem}              % For customizing lists

% ---------------------------------------------------------------------------
% Custom Commands and Macros
% ---------------------------------------------------------------------------
\newcommand{\R}{\mathbb{R}}
\newcommand{\norm}[1]{\left\lVert #1 \right\rVert}

% ---------------------------------------------------------------------------
% Document Metadata
% ---------------------------------------------------------------------------
\title{A More Comprehensive \LaTeX~Document}
\author{Your Name}
\date{\today}

\begin{document}

\maketitle

% ---------------------------------------------------------------------------
% Section 1: Introduction
% ---------------------------------------------------------------------------
\section{Introduction}
This document showcases a variety of \LaTeX\ features:
\begin{itemize}
    \item Sectioning and cross-referencing
    \item Text formatting and environments
    \item Figures, tables, and math modes
    \item Bibliography management 
\end{itemize}

As a quick example, let's cite \cite{einstein1905} to demonstrate how bibliographic references work.
Then we can print the reference list with \texttt{\textbackslash printbibliography} at the end.

% ---------------------------------------------------------------------------
% Section 2: Text Formatting
% ---------------------------------------------------------------------------
\section{Text Formatting}
We can highlight \textbf{bold text}, produce \textit{italicized text}, or \underline{underlined text}.
\noindent We also have special characters like \% (percent sign) which we escape with a backslash.

\subsection{Lists and Environments}
\paragraph{Unordered List:}
\begin{itemize}
    \item First item
    \item Second item
\end{itemize}

\paragraph{Ordered List (with custom labels):}
\begin{enumerate}[label=(\roman*)]
    \item First item
    \item Second item
\end{enumerate}

\paragraph{Description Environment:}
\begin{description}
    \item[LaTeX] A typesetting system.
    \item[Python] A programming language.
\end{description}

% ---------------------------------------------------------------------------
% Section 3: Mathematical Environments
% ---------------------------------------------------------------------------
\section{Mathematics in \LaTeX}
\subsection{Inline Math}
This is inline math: $ x = 2 $.

\subsection{Display Math}
\[
   x = \frac{-b \pm \sqrt{b^2 - 4ac}}{2a}.
\]
Numbered equations:
\begin{equation}\label{eq:quad}
   x = \frac{-b \pm \sqrt{b^2 - 4ac}}{2a}.
\end{equation}
And in Eq.~\ref{eq:quad}, we see the quadratic formula.

\subsection{Multiple Equations}
\begin{align}
   x &= 2, \\
   y &= 3.
\end{align}
You can switch to \texttt{align*} to suppress numbering.

\subsection{Custom Macros}
Using our custom commands:
\[
   \norm{x} \in \R.
\]

% ---------------------------------------------------------------------------
% Section 4: Figures and Tables
% ---------------------------------------------------------------------------
\section{Figures and Tables}

\subsection{Figure Example}
\begin{figure}[ht]
    \centering
    % Replace with your own image path or keep the placeholder
    \includegraphics[width=0.3\textwidth]{example-image-a}
    \caption{A placeholder figure.}
    \label{fig:placeholder}
\end{figure}
We can reference Figure~\ref{fig:placeholder} easily.

\subsection{Table Example}
\begin{table}[ht]
    \centering
    \begin{tabular}{c c c}
    \toprule
    A & B & C \\
    \midrule
    1 & 2 & 3 \\
    4 & 5 & 6 \\
    \bottomrule
    \end{tabular}
    \caption{A simple table.}
    \label{tab:mytable}
\end{table}
We can reference Table~\ref{tab:mytable} here.

% ---------------------------------------------------------------------------
% Section 5: Example Content
% ---------------------------------------------------------------------------
\section{Example Content}
\subsection{Sample Text}
\lipsum[1]  % Generates some sample paragraph text
For further information, see \cite{einstein1905}.

% ---------------------------------------------------------------------------
% Section 6: Bibliography
% ---------------------------------------------------------------------------
\section{References}
\printbibliography

\end{document}
